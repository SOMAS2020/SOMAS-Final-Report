\chapter{Game Design}

\section{Implementation}
\label{sec:GD:implementation}

Implementation design was important as the majority of the class were involved in writing simulation code. As such, dedicated \emph{infrastructure} engineers from each team were elected to form an infrastructure team responsible for building the central part of the game, otherwise known as the \emph{server}. Each team's infrastructure engineer was tasked to own the implementation of a slice of the server.

One such sub-team was the \textbf{core} infrastructure team, responsible for building the core parts of the game server. Inputs were taken from the entire class to choose the best implementation strategies and design--a solid and simple core foundation was paramount to allow clean continuous integration of code and ideas from all contributors. The subsections below detail implementation specifics of the game's simulation.

\subsection{Architecture}
\label{sec:GD:implementation:arch}

The structure of the game was closely modelled after a \emph{client-server} model\footnote{\url{https://en.wikipedia.org/wiki/Client-server_model}}, but note ``closely''--whilst the nomenclature was taken directly from the aforementioned model, there are some differences. \emph{Server} and \emph{client} in this context bear the following definitions:

\begin{definition} \label{def:server}
    The \textbf{server} is the central game runner, responsible for initiating game events (such as a disaster or the start of a turn). Certain events require the actions of agents, in which the server will invoke a function on the agents to receive a response. Further, the server acts as a source-of-truth for the game's state.
\end{definition}


\begin{definition} \label{def:client}
    Each \textbf{client} is implemented by an agent. The client provides an interface of functions in which the server can invoke. Moreover, clients may also invoke certain functions from the server's interface to read specific game information.
\end{definition}

The major difference of this architecture to a traditional client-server model is that, in the former, the stateful central server drives the other clients (by triggering events and eliciting responses), as opposed to clients sending stateless requests to a server to receive a response in the latter. Furthermore, whilst the client and server have been separated architecturally, the entire system still operates as a single process.

\subsection{Technology Stack}
\label{sec:GD:implementation:techstack}

Agreeing on a technology stack proved to be challenging--the class had varying levels of programming expertise. Whilst Prolog\footnote{\url{https://en.wikipedia.org/wiki/Prolog}} and Qu-Prolog\footnote{\url{https://staff.itee.uq.edu.au/pjr/HomePages/QuPrologHome.html}} (an extension to the former) were used to cover topics in the lectures, the class did not favour them over more well-known and established imperative programming languages. Hence, time was dedicated to formulate a consensus to decide on the technology stack, with primary focus given on the choice of implementation language. Firstly, high priority requirements were defined for the stack:

\begin{itemize}
    \item Easy to learn
    \item Easy to setup
    \item Cross-platform
    \item Easily maintainable
    \item Friendly features
\end{itemize}

Minimum Working Examples (MWEs) comprising a single server and two clients were created in different stacks, which served as starting points for discussion among members of the class. The MWEs implemented are as follows.

\begin{enumerate}
    \item \textbf{Multi-language}\footnote{\url{https://github.com/SOMAS2020/somas-demo}}.
          A multi-language (C++ server with a Python and a C++ client) MWE was first set up. It was first thought that allowing agent teams to choose the programming language they were most familiar with would speed up development. However, despite this stack's benefits, it could not be easily made cross-platform. Each agent's code would need to be run in a separate process, and Inter-Process Communication (IPC) would be required to pass messages. IPC is quite low-level and varies vastly among different systems. Further, protocols for the IPC would also need to be set up to pass the correct message, and having no strong typing as in a strongly-typed language (or session-typed\footnote{\url{https://arxiv.org/abs/1906.03836}}) approach would make it difficult to maintain and develop.

    \item \textbf{Python}\footnote{\url{https://github.com/SOMAS2020/somas-demo-py}}.
          Python\footnote{\url{https://www.python.org/}} is widely used by the scientific community in recent years with the growing ubiquity of scientific computation packages available for it. As such, Python was a strong contender as most of the class already know Python from past projects. However, Python's weak typing meant it scored low on the ``easily maintainable'' part--the server and client interfaces would benefit a lot from strong typing. While add-on static typing tools such as \texttt{mypy}\footnote{\url{https://mypy.readthedocs.io/}} could be employed (it is also used in the MWE), it would still not be as powerful as built-in strong typing as in languages such as C\# and Go.

    \item \textbf{C\#}\footnote{\url{https://github.com/SOMAS2020/somas-demo-cs}}.
          C\#\footnote{\url{https://docs.microsoft.com/en-us/dotnet/csharp/}} is the flagship language of the .NET ecosystem. C\# shares a large part of its design to the more popular C++. C\# is strongly-typed, and promotes use of clean Object-Oriented Programming (OOP). Many of the features from C/C++ that can be \emph{dangerous} are not present or hidden, making it more beginner friendly. A drawback is that some experience with C-family languages is required to pick up C\# quickly.

    \item \textbf{Go}\footnote{\url{https://github.com/SOMAS2020/somas-demo-go}}.
          While most of the class was not familiar with Go\footnote{\url{https://golang.org/}}, its simple language syntax and highly-featured toolchain make it very easy to learn. The modern Go toolchain makes it extremely simple for programs to work cross-platform. While Go's omission of OOP and generics might be seen as a disadvantage, it makes it an easy language to learn, and prevents pitfalls commonly caused by such ``features''. Go, like C\#, is strongly typed. Moreover, Golang's great support for WebAssembly\footnote{\url{https://webassembly.org/}} would prove useful for visualisations, further detailed in~\ref{sec:GD:implementation:visualisations}. Another nice feature is that concurrency can be easily implemented in Go, which meant that agent actions could be run concurrently to speed up simulations.
\end{enumerate}

After discussion, scores (out of 10) were given for each stack. Table~\ref{table:techstackscores} shows these scores. The multi-language and Python approaches were removed from consideration--the former due to its low total score and the latter because of its low maintainability. The decision between C\# and Go was harder, and ultimately resulted in a \emph{simple majority} vote by the class. 37 people voted in total, with 25 in favour of Go. Hence, Go was finally chosen.

\begin{table}[h]
    \centering
    \caption{Scores given for each stack based on requirements}
    \label{table:techstackscores}
    \begin{tabular}{|c|c|c|c|c|c|c|}
        \hline
        Stack                     &
        \makecell{Easy              \\ to \\ learn}              &
        \makecell{Easy              \\ to \\ setup}              &
        \makecell{Cross-platform} &
        \makecell{Easily            \\ maintainable}             &
        \makecell{Friendly          \\ features}                 &
        \makecell{Final             \\ score \\ (out of 50)}
        \\
        \hline
        Multi-language            &
        6                         &
        2                         &
        0                         &
        2                         &
        10                        &
        20
        \\
        \hline
        Python                    &
        8                         &
        6                         &
        8                         &
        3                         &
        8                         &
        34
        \\
        \hline
        C\#                       &
        6                         &
        5                         &
        8                         &
        9                         &
        8                         &
        37
        \\
        \hline
        Go                        &
        8                         &
        8                         &
        10                        &
        8                         &
        6                         &
        40
        \\
        \hline
    \end{tabular}
\end{table}


\subsection{Visualisations}
\label{sec:GD:implementation:visualisations}

\subsubsection{Toolchain}
A benefit from the choice of the Go technology stack was that it supports compiling source code into WebAssembly out of the box. WebAssembly can be run efficiently in most modern browsers, which meant that in-browser simulations can be run and then visualised on a website.

The website (\url{https://somas2020.github.io/SOMAS2020/}) features... % TODO:- Vis team: yp717 et al.

\subsection{Engineering Practices}
\label{sec:GD:implementation:practices}

Developing and maintaining a codebase with contributions from around 40 developers was projected to be non-trivial. Therefore, good software engineering practices and rules were employed to be upheld by all contributors to make the process smoother and--where possible--automated.

\subsubsection{Code testing}

While Go is strongly typed and would mean that most errors can be detected at compile-time (or even at time of coding with its performant language server), having unit and integration tests greatly improved the maintainability and ease of development of the project--these issues were particularly important as this was a large-scale group software engineering project--developers stepping on each other's toes is a common occurrence in non-tested group project code. Henceforth, tests were required for non-trivial server-side code.

\subsection{Peer Review}

Peer review was also setup via GitHub pull requests--each change required the approval of another member in the infrastructure team. This practice not only promoted consistent code design and implementation, it minimised mistakes and ensured that the code implementation meets design and implementation requirements set forth. Further, as there was variation in programming skill among code contributors, knowledge sharing was facilitated by code reviews. This was also a critical opportunity for engineers to learn more about the other code being contributed to the project.

\subsection{Continuous Integration}
\label{sec:GD:implementation:practices:CI}

Via GitHub Actions\footnote{https://github.com/features/actions}, continuous integration was set up. Each Pull Request (PR) was set up to trigger automated runs of written tests and a full simulation in addition to static code analysers such as linters. These checks must all pass for the PR to be merged into the main branch. Automated testing and code analysis saved time and facilitates regression testing, as all tests--existing and new--were run for each change. Running a full simulation also served as a good stress-test for the system to make sure that it does not crash on a similar full simulation. Further, automated tests were run on a reference system (Ubuntu 20.04 with Go 1.15.5 and Node 14), helping to prevent system-specific quirks or bugs from polluting the codebase.

\subsection{Continuous Development}

On receiving PR approval, passing automated tests and finally merging into the main codebase, the visualisation website is automatically rebuilt with the latest changes. This saved time as manual builds were not required. The builds were produced on a reference system (identical to that mentioned in~\ref{sec:GD:implementation:practices}), ensuring consistent builds free from system-specific quirks. Further, automated builds meant that the website always runs on the latest codebase.
